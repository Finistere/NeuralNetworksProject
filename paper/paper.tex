\documentclass[twoside,11pt]{article}

% Any additional packages needed should be included after jmlr2e.
% Note that jmlr2e.sty includes epsfig, amssymb, natbib and graphicx,
% and defines many common macros, such as 'proof' and 'example'.
%
% It also sets the bibliographystyle to plainnat; for more information on
% natbib citation styles, see the natbib documentation, a copy of which
% is archived at http://www.jmlr.org/format/natbib.pdf

\usepackage{jmlr2e}

% Definitions of handy macros can go here

\newcommand{\dataset}{{\cal D}}
\newcommand{\fracpartial}[2]{\frac{\partial #1}{\partial  #2}}

% Heading arguments are {volume}{year}{pages}{submitted}{published}{author-full-names}

%\jmlrheading{1}{2000}{1-48}{4/00}{10/00}{Marina Meil\u{a} and Michael I. Jordan}

% Short headings should be running head and authors last names

%\ShortHeadings{Ensemble Methods for Robust Feature Selection}{Meil\u{a} and Jordan}
\firstpageno{1}

\begin{document}

\title{Ensemble Methods for Robust Feature Selection}

\author{\name Marina Meil\u{a} \email mmp@stat.washington.edu \\
       \addr Department of Statistics\\
       University of Washington\\
       Seattle, WA 98195-4322, USA
       \AND
       \name Michael I.\ Jordan \email jordan@cs.berkeley.edu \\
       \addr Division of Computer Science and Department of Statistics\\
       University of California\\
       Berkeley, CA 94720-1776, USA}

\editor{Leslie Pack Kaelbling}

\maketitle

\begin{abstract}%   <- trailing '%' for backward compatibility of .sty file
  Abstract
\end{abstract}

\begin{keywords}
  Ensemble Methods, Robust, 
\end{keywords}

\section{Motivation}

Features selection methods is a key method to work with high dimensional data.
It is widely used in areas working with high dimensional data like bioinformatics.
In the past different feature selection methods have been presented.
Additionally, ensemble methods have been presented by \cite{saeys2008}

% Acknowledgements should go at the end, before appendices and references

%\acks{I want to thank my mum}

% Manual newpage inserted to improve layout of sample file - not
% needed in general before appendices/bibliography.

\newpage

\appendix
%\section*{Appendix A.}
%\label{app:some appendix}

% Note: in this sample, the section number is hard-coded in. Following
% proper LaTeX conventions, it should properly be coded as a reference:

\vskip 0.2in
\bibliography{references}

\end{document}
