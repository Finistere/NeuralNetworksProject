\documentclass[twoside,11pt]{article}

% Any additional packages needed should be included after jmlr2e.
% Note that jmlr2e.sty includes epsfig, amssymb, natbib and graphicx,
% and defines many common macros, such as 'proof' and 'example'.
%
% It also sets the bibliographystyle to plainnat; for more information on
% natbib citation styles, see the natbib documentation, a copy of which
% is archived at http://www.jmlr.org/format/natbib.pdf

\usepackage{jmlr2e}
\usepackage{amsmath}

% Definitions of handy macros can go here

\newcommand{\dataset}{{\cal D}}
\newcommand{\fracpartial}[2]{\frac{\partial #1}{\partial  #2}}

% Heading arguments are {volume}{year}{pages}{submitted}{published}{author-full-names}

%\jmlrheading{1}{2000}{1-48}{4/00}{10/00}{Marina Meil\u{a} and Michael I. Jordan}

% Short headings should be running head and authors last names

%\ShortHeadings{Ensemble Methods for Robust Feature Selection}{Meil\u{a} and Jordan}
\firstpageno{1}

\begin{document}

\title{Ensemble Methods for Robust Feature Selection}

\author{\name Marina Meil\u{a} \email mmp@stat.washington.edu \\
       \addr Department of Statistics\\
       University of Washington\\
       Seattle, WA 98195-4322, USA
       \AND
       \name Michael I.\ Jordan \email jordan@cs.berkeley.edu \\
       \addr Division of Computer Science and Department of Statistics\\
       University of California\\
       Berkeley, CA 94720-1776, USA}

\editor{Leslie Pack Kaelbling}

\maketitle

\begin{abstract}%   <- trailing '%' for backward compatibility of .sty file
  Abstract
\end{abstract}

\begin{keywords}
  Ensemble Methods, Robust, 
\end{keywords}

\section{Introduction}

Feature selection is a preprocessing step used in machine learning application to find
a small subset of features in order to build a more accurate, simpler and faster model.
Moreover it allows domain experts to gain a better understanding of the data, by focusing
on the selected subset.However in domains such as biomedical applications, 
where subsequent analysis are costly, not only the model performance but also the robustness
of the feature selection method is important. Domain experts would prefer a more stable
algorithm to have more confidence in the selected features. One approach for more robust
results are ensemble methods.

In ensemble learning multiple algorithms are combined to obtain better performance or stability.
We build on the work of \cite{saeys2008} who applied ensemble learning by using
the same feature selector on different subsets. In this work we investigate the effect of ensemble methods
further by using multiple feature selection methods. 
In order to find a good trade-off between robustness and model performance, multiple feature selection methods were
combined in multiple ways.

\section{Feature selection methods}

Symmetrical Uncertainty \citep{press1996numerical} is entropy based measurement. It is the normalized mutual information between the feature
and the class. The mutual information measures the influence of the knowledge of the class on the feature value
in terms of entropy and normalizes it. More details about mutual information can be found in \cite{paninski2003estimation}.
\begin{equation}
  \label{eq:su}
  SU(F,C) = 2 \frac{H(F) - H(F|C)}{H(F) + H(C)} \textrm{, where } H(\cdot) \textrm{ is the entropy.}
\end{equation}

The second feature selection method we used is RELIEF \citep{kira1992feature}. In the beginning the weights are initialize
with a $0$ weight.
It then iteratively takes a random sample $x_i$ and compares it to their nearest neighboring sample belonging to their class
$\textrm{Near-hit}_i$ and the opposite one $\textrm{Near-miss}_i$ in terms of a similarity measurement.
The weights are set by how well these two similarity measurements distinguish as it can be seen in equation \ref{eq:relief}.
In the standard case of RELIEF for comparisment the euclidean distance is used for numerical values. This inherits the
assumption that features which are nearer to each other are also more similar, which should not be correct in each application
area.
\begin{equation}
  \label{eq:relief}
  W_i = W_i - \|x_i - \textrm{Near-hit}_i\|_2^2 + \|x_i - \textrm{Near-miss}_i\|_2^2
\end{equation}

The third feature selection method we used was a recursive feature elmination in combination with SVM \citep{guyon2002gene}.
Recursive feature elimination fits a SVM and ranks the feature according to their importance in the model,
based on the weight vector of the hyperplane.
More precisely after obtaining the solution for the primal problem of the SVM in equation \ref{eq:svm_primal},
the ranking coefficients $c_i$ for the features are then calculated by squaring the weights $c_i = w_i^2$.
\begin{alignat}{-1}
     \min_{w,b}  & \quad \frac12\|w\|^2 + C\sum_{i=1}^m \xi_i & & & \label{eq:svm_primal} \\ 
   \text{s.\,t.} & \quad y^{(i)} (w^Tx^{i} + b) & \quad \geq & \quad 1 -\xi_i &&
                   \quad \forall \, i = 1, \ldots, m \nonumber \\
                 & \quad \xi_i                  & \quad \geq & \quad 0 &&
                   \quad \forall \, i = 1, \ldots, m \nonumber 
\end{alignat}

The worst ranked features are then discarded and the procedure is repeated until only a subset of features is left.
In our case we discarded each step $10\%$ of the size of all features of the features being left. This means that
in each step we discarded the same amount of features. This procedure was repeated until a subset of the size
of $1\%$ of all features was left.

\section{Ensemble methods}
%Weighting
We first calculate the weights if avaible by the features selector otherwise the ranks.
For feature selector, which do not omit normalized weights between a0 and 1, we normalize
the weights to the range $[0,1]$. We also normalize the ranks of the features selectors returning
ranks to the range $[0,1]$. Now that we combine the weights after one of the following rules: \ldots

Ensemble method: The weights of different feature selection methods are first computed and then linearly aggregated to obtain the new weights. Three methods were used:
The mean is used as the new weight.
The minimum, maximum and the mean are linearly aggregated.
The accuracy is estimated for each of the feature selector for multiple classifiers. Then the different feature weights are accordingly weighted to favor the feature selector with the best performance and the previous method is applied.


\section{Results}
- Do SVM\_RFE with exponential rising
- SU achieved high values in the paper, in our results SU weighted a lot of features the 
same; therefore for robustness same weights have to be shuffled to not accidental increase
robustness by order of features
- The noise features of the NIPS 2003 challenge data sets are highly correlated with the
real feature; therefore no improvement has been made with feature selection in terms of 
accuracy
% Acknowledgements should go at the end, before appendices and references

%\acks{I want to thank my mum}

% Manual newpage inserted to improve layout of sample file - not
% needed in general before appendices/bibliography.

\newpage

\appendix
%\section*{Appendix A.}
%\label{app:some appendix}

% Note: in this sample, the section number is hard-coded in. Following
% proper LaTeX conventions, it should properly be coded as a reference:

\vskip 0.2in
\bibliography{references}

\end{document}
